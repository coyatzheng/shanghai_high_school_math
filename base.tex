\usepackage{xparse}


\usepackage{newtxtext}
\usepackage{geometry}
\usepackage{lipsum} % 该宏包是通过 \lipsum 命令生成一段本文,正式使用时不需要引用该宏包
\usepackage[dvipsnames,svgnames]{xcolor}
\usepackage[strict]{changepage} % 提供一个 adjustwidth 环境
\usepackage{framed} % 实现方框效果

\def\TeacherName{郑老师}

% environment derived from framed.sty: see leftbar environment definition
\definecolor{formalshade}{rgb}{0.95,0.95,1} % 文本框颜色
% ------------------******-------------------
% 注意行末需要把空格注释掉,不然画出来的方框会有空白竖线
\newenvironment{formal}{%
\def\FrameCommand{%
\hspace{1pt}%
{\color{DarkBlue}\vrule width 2pt}%
{\color{formalshade}\vrule width 4pt}%
\colorbox{formalshade}%
}%
\MakeFramed{\advance\hsize-\width\FrameRestore}%
\noindent\hspace{-4.55pt}% disable indenting first paragraph
\begin{adjustwidth}{}{7pt}%
\vspace{2pt}\vspace{2pt}%
}
{%
\vspace{2pt}\end{adjustwidth}\endMakeFramed%
}
% ------------------******-------------------


\newcommand{\classinfo}[4]{
\begin{center}
    \begin{tabular}{ | m{5em}<{\centering} | m{4em}<{\centering}| m{4em}<{\centering} | m{4em}<{\centering} | m{5em}<{\centering} | m{5em}<{\centering} | m{4em}<{\centering} | m{5em}<{\centering} | } 
    \hline
    教师姓名& \TeacherName & 学生姓名&   & 等级& #1 & 上课时间 & \\ 
    \hline
    学\quad\quad 科& 高中数学 & 课题名称 &  \multicolumn{5}{c|}{#2} \\ 
    \hline
    教学目标 & \multicolumn{7}{l|}{#3} \\ 
    \hline
    教学重难点 & \multicolumn{7}{l|}{#4} \\ 
    \hline
    \end{tabular}
 \end{center}
}



\newif\ifprint

\newcommand{\tiankong}[1]{\ \underline{
\ifprint
\ #1
\else
\hspace*{5em}
\fi}}

\newif\ifjiexi
\NewDocumentEnvironment{jiexi}{ +b }{
\ifjiexi
\par
{\bfseries 解析}\, #1
\else
{\vspace{2cm}}
\fi
}{\par}

\newif\ifshowanswer
\showanswertrue %答案控制看这里

\ifshowanswer
\printtrue
\jiexitrue
\else
\printfalse
\jiexifalse
\fi 