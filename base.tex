\usepackage{xparse}


\usepackage{newtxtext}
\usepackage{geometry}
\usepackage{lipsum} % 该宏包是通过 \lipsum 命令生成一段本文,正式使用时不需要引用该宏包
\usepackage[dvipsnames,svgnames]{xcolor}
\usepackage[strict]{changepage} % 提供一个 adjustwidth 环境
\usepackage{framed} % 实现方框效果

\def\TeacherName{郑振宇}


\usepackage{draftwatermark}         % 所有页加水印
%\usepackage[firstpage]{draftwatermark} % 只有第一页加水印
\SetWatermarkText{郑振宇}           % 设置水印内容
%\SetWatermarkText{\includegraphics{fig/texlion.png}}         % 设置水印logo
\SetWatermarkLightness{0.94}             % 设置水印透明度 0-1
\SetWatermarkScale{0.4}                   % 设置水印大小 0-1  
%

%--------------------------------------选择题---------------------------
\usepackage{ifthen}
\newlength{\la}
\newlength{\lb}
\newlength{\lc}
\newlength{\ld}
\newlength{\lhalf}
\newlength{\lquarter}
\newlength{\lmax}
\newcommand{\fourchoice}[4]{  
\settowidth{\la}{A.~#1~~~}  
\settowidth{\lb}{B.~#2~~~}  
\settowidth{\lc}{C.~#3~~~}  
\settowidth{\ld}{D.~#4~~~}  
\ifthenelse{\lengthtest{\la > \lb}}  
{\setlength{\lmax}{\la}}  
{\setlength{\lmax}{\lb}}  
\ifthenelse{\lengthtest{\lmax < \lc}}  
{\setlength{\lmax}{\lc}}  {}  
\ifthenelse{\lengthtest{\lmax < \ld}} 
{\setlength{\lmax}{\ld}}  {}  
\setlength{\lhalf}{0.5\linewidth}  
\setlength{\lquarter}{0.25\linewidth}  
\ifthenelse{\lengthtest{\lmax > \lhalf}} 
{\noindent{}A.~#1 \\ B.~#2 \\ C.~#3 \\ D.~#4 }  
{  \ifthenelse{\lengthtest{\lmax > \lquarter}}  
{\noindent\makebox[\lhalf][l]{A.~#1~~~}%    
\makebox[\lhalf][l]{B.~#2~~~}%    
\makebox[\lhalf][l]{C.~#3~~~}%    
\makebox[\lhalf][l]{D.~#4~~~}}%    
{\noindent\makebox[\lquarter][l]{A.~#1~~~}%      
\makebox[\lquarter][l]{B.~#2~~~}%      
\makebox[\lquarter][l]{C.~#3~~~}%      
\makebox[\lquarter][l]{D.~#4~~~}}}}
%-----------------------------------------------------------------------




% environment derived from framed.sty: see leftbar environment definition
\definecolor{formalshade}{rgb}{0.95,0.95,1} % 文本框颜色
% ------------------******-------------------
% 注意行末需要把空格注释掉,不然画出来的方框会有空白竖线
\newenvironment{formal}{%
\def\FrameCommand{%
\hspace{1pt}%
{\color{DarkBlue}\vrule width 2pt}%
{\color{formalshade}\vrule width 4pt}%
\colorbox{formalshade}%
}%
\MakeFramed{\advance\hsize-\width\FrameRestore}%
\noindent\hspace{-4.55pt}% disable indenting first paragraph
\begin{adjustwidth}{}{7pt}%
\vspace{2pt}\vspace{2pt}%
}
{%
\vspace{2pt}\end{adjustwidth}\endMakeFramed%
}
% ------------------******-------------------


\newcommand{\classinfo}[4]{
\begin{center}
    \begin{tabular}{ | m{5em}<{\centering} | m{4em}<{\centering}| m{4em}<{\centering} | m{4em}<{\centering} | m{5em}<{\centering} | m{5em}<{\centering} | m{4em}<{\centering} | m{5em}<{\centering} | } 
    \hline
    教师姓名& \TeacherName & 学生姓名&   & 等级& #1 & 上课时间 & \\ 
    \hline
    学\quad\quad 科& 高中数学 & 课题名称 &  \multicolumn{5}{c|}{#2} \\ 
    \hline
    教学目标 & \multicolumn{7}{l|}{#3} \\ 
    \hline
    教学重难点 & \multicolumn{7}{l|}{#4} \\ 
    \hline
    \end{tabular}
 \end{center}
}



\newif\ifprint

\newcommand{\xuanze}[1]{\ (
\ifprint
\ #1
\else
\hspace*{4em}
\fi)}

\newcommand{\tiankong}[1]{\ \underline{
\ifprint
\ #1
\else
\hspace*{5em}
\fi}}

\newif\ifjiexi
\NewDocumentEnvironment{jiexi}{ +b }{
\ifjiexi
\par
{\bfseries 解析}\, #1
\else
{\vspace{2cm}}
\fi
}{\par}

\newif\ifshowanswer
\showanswertrue %答案控制看这里

\ifshowanswer
\printtrue
\jiexitrue
\else
\printfalse
\jiexifalse
\fi 