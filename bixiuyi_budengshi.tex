% main.tex
\documentclass[10pt,a4paper]{ctexbook}
\usepackage{makeidx} % 调用 makeidx 宏包,用来处理索引
\usepackage{tabto}
\usepackage{tikz}                   % 绘制图形
\usepackage{tikz-3dplot}            % 绘制三维坐标系,坐标变换
\usepackage{siunitx}
% \makeindex % 开启索引的收集

\title{沪教版高中数学讲义} 
\author{ 郑振宇\thanks{邮箱:coyatzheng@gmail.com}}
\date{\today}


\begin{document}
\setlength{\parindent}{0em}


\maketitle



\chapter{集合与逻辑}
ssss
\chapter{等式和不等式}

\section{等式的性质与方程的解集}
\begin{enumerate}
    \item 等式的性质
    \begin{enumerate}
        \item 传递性 \NumTabs{16} 设$a,b$均为实数,如果$a=b$,且$b=c$,那么$a=c$ 
        \item 加法性质 \NumTabs{16} 设$a,b$均为实数,那么$a+c=b+c$ 
        \item 乘法性质 \NumTabs{16} 设$a,b$均为实数,那么$ac=bc$ 
    \end{enumerate}
    \item 方程的解集 \quad  含有未知数的等式称为方程.使得方程两端相等的未知数的值,称为方程的解或者方程的根。
    以方程的解为元素所构成的集合称为方程的解集。
\end{enumerate}

\section{一元二次方程的解集及根与系数的关系}

1、概念:形如 $ax^2+bx+c,(a \ne 0)$ 的方程为一元二次方程;\par
2、配方法:对一元二次方程进行配方得到方程:$(x + \frac{b}{2a})^2 = \frac{b^2-4ac}{4a^2} $ \par
3、判别式$\Delta$ \par
从配方之后的方程可以看出:原方程有没有解,取决于代数式$ b^2-4ac $的正负;基于$ b^2-4ac $的重要性,令称为该一元二次方程的判别式,它决定了一元二次方程解的个数问题;\par
(1)若$\Delta > 0$,原方程有两个不等的实数根,这两个根是$ x_1 = \frac{-b+\sqrt{b^2-4ac}}{2a} , x_2 = \frac{-b-\sqrt{b^2-4ac}}{2a}$;\par
(2)若$\Delta = 0$,原方程有两个相等的实数根,$ x_1 = x_2 = -\frac{b}{2a}$;\par
(3)若$\Delta < 0$,原方程没有实根;\par
4、韦达定理\par
当上述一元二次方程有实数解时,$ x_1 = \frac{-b+\sqrt{b^2-4ac}}{2a} , x_2 = \frac{-b-\sqrt{b^2-4ac}}{2a}$ \par
则:$$ x_1+x_2=\frac{-b+\sqrt{\Delta}}{2a}+\frac{-b-\sqrt{\Delta}}{2a}=-\frac{b}{a}$$
$$ x_1 \cdot x_2=\frac{-b+\sqrt{b^2-4ac}}{2a} \cdot \frac{-b-\sqrt{b^2-4ac}}{2a}=\frac{c}{a}$$

注意:在现在所学范围下,使用韦达定理时,需判别$\Delta = b^2-4ac \ge 0 $. \par
特别的:
$$x_1^2+x_2^2 = (x_1+x_2)^2-2x_1x_2 \qquad \qquad \frac{1}{x_1}+\frac{1}{x_2} = \frac{x_1+x_2}{x_1x_2}$$
$$(x_1-x_2)^2 = (x_1+x_2)^2-4x_1x_2 \qquad \qquad | x_1-x_2 | = \sqrt{(x_1+x_2)^2-4x_1x_2}$$
一元二次方程的两根之差的绝对值$| x_1-x_2 | = \frac{\sqrt{\Delta}}{|a|}$



\section{不等式的求解1}
一元一次不等式组\par

\end{document}